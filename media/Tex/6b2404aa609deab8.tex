\documentclass[preview]{standalone}
\usepackage[english]{babel}
\usepackage{amsmath}
\usepackage{amssymb}
\begin{document}
\begin{center}
A more formal discussion of the limitations of physical properties involves recognizing that every property has a representation as a subspace of some extended Hilbert space, with either a finite or uncountably infinite set of basis vectors called eigenvectors, each associated with an eigenvalue. Each eigenvalue represents a physically measurable quantity.

            Of course, this is the kind of topic that could occupy half a semester in a rigorous quantum mechanics course. Though we will briefly talk about eigenvectors and eigenvalues later, I won't delve too deeply into this formalism.
\end{center}
\end{document}